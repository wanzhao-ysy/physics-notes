% 曲面及面元矢量示意图
\begin{figure}[htbp]
  \centering
  \begin{tikzpicture}[x=1cm, y=1cm, z=-0.6cm, scale=0.6]
    % 1. 绘制曲面 S (用平滑曲线围成一个类四边形曲面)
    % 定义四个角点的大致位置,利用 tension 控制弯曲
    \draw [thick, fill=gray!10] plot [smooth cycle, tension=0.7] coordinates {
      (0,0) (5,1) (6,5) (1,4)
    };
    
    % 2. 标注 内、外、S
    % 假设曲面是封闭曲面的一部分,或者是开放曲面区分两侧
    % 根据描述“左边内,右边外”,这里简单放置文本
    \node at (-0.5, 2.5) {内};
    \node at (6.5, 2.5) {外};
    \node at (2, 1.5) {\Large $S$};

    % 3. 绘制面元 dS (一个小的平行四边形)
    \begin{scope}[shift={(3.8, 3.2)}, rotate=-10] 
        % 在坐标 (3.8,3.2) 附近画一个小平行四边形
        % 定义面元的四个顶点,模拟透视
        \coordinate (p1) at (0,0);
        \coordinate (p2) at (0.6, 0.1);
        \coordinate (p3) at (0.7, 0.7);
        \coordinate (p4) at (0.1, 0.6);
        
        \draw[fill=white, thin] (p1) -- (p2) -- (p3) -- (p4) -- cycle;
        
        % 面元标注 dS
        \node[below right, font=\footnotesize] at (p2) {$\d \bm{S}$};

        % 4. 绘制法向量和面元矢量
        % 中心点
        \coordinate (center) at (0.35, 0.35);
        
        % 法向矢量 n (稍长)
        \draw[->, thick, blue] (center) -- ++(70:1.5) node[above] {$\bm{n}$};
        
        % 面元矢量 dS vector (稍短,方向相同)
        \draw[->, thick, red] (center) -- ++(70:0.8) node[right, font=\footnotesize, xshift=2pt] {$\d \bm{S} = \d S \bm{n}$};
    \end{scope}

  \end{tikzpicture}
  \caption{}
  \label{fig:surface-element}
\end{figure}

