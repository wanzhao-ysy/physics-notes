\chapter{数学基础}\label{chap:math-basics}
\addtocontents{los}{\protect\addvspace{10pt}}

\begin{intro}
物理的学习离不开数学,但学习时间的有限使得很多同学纠结数学应该学到多深。在普通物理的范畴内,作者在这里提出一些小小的建议。

数学知识分两类。

第一类是需要完全理解和掌握的,比如数列、导数和积分、微分方程……它们在题目中出现的概率很高,当我们遇到时要能从容应对。

第二类是有助于物理理解,但不要求完全掌握的,比如矢量分析与场论、张量代数……他们有助于加深你对物理情景的理解,但在普通物理的题目中出现的概率较低,这种知识需要学习,但学习的深度要视自身情况而定。
\end{intro}

\section{一元函数微积分}\label{sec:single-variable-calculus}

\subsection{微分}\label{subsec:differential}

一元函数可记为
\begin{equation}
    y = y(x)\ \text{或}\ y = f(x),
\end{equation}

在它的连续区域内,如图

\begin{tikzpicture}
    \draw[->] (-0.5,0) -- (4,0) node[right] {$x$};
    \draw[->] (0,-0.5) -- (0,4) node[above] {$y$};
    
    \node[below left] at (0,0) {$O$};
\end{tikzpicture}

